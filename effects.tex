\documentclass[preprint,9pt,numbers]{sigplanconf}

%% Packages
\usepackage[utf8]{inputenc}   % Use UTF-8 encoding
\usepackage[english]{babel}   % Enforce English language settings
\usepackage{hyperref}         % Interactive PDF
\usepackage[T1]{fontenc}      % Fix font rendering
\usepackage{times}            % Use times font
%\usepackage[numbers]{natbib}  % Bibliography
\usepackage{mathtools}        % Maths typesetting
\usepackage{mathpartir}       % Inference rules
\usepackage{mathwidth}        % Render character sequences nicely in math mode
\usepackage{xspace}           % proper spacing for macros in text
\usepackage{tikz}             % Drawing pictures
\usepackage[scaled]{beramono} % Typewriter font
\usepackage{flushend}         % Balance columns

%%
%% Convenient macros
%%

% TODOs
\newcommand{\todo}[1]{{\par\noindent\small\color{red} \framebox{\parbox{\dimexpr\linewidth-2\fboxsep-2\fboxrule}{\textbf{TODO:} #1}}}}

%% E-mail obfuscation
\newcommand{\reachme}[1]{\href{mailto:#1@ed.ac.uk}{\MakeLowercase{#1@ed.ac.uk}}}

%% Metainformation

\newcommand{\thetitle}{The State of Effectful Programming}
\newcommand{\theauthor}{Daniel Hillerström}
\newcommand{\theaffiliation}{\small{LFCS, The University of Edinburgh}}
\newcommand{\theemail}{\small{\reachme{daniel.hillerstrom}}}

\title{\thetitle{}}
% \author{\theauthor{}\\\theaffiliation{}\\\theemail{}}
% \date{\small{\today}}

\authorinfo{\theauthor{}}
{\theaffiliation{}}
{\theemail{}}

%% Begin document
\begin{document}

%% Remove SIGPLANCONF copyright space
\makeatletter
\def\@copyrightspace{\relax}
\makeatother

\titlebanner{DRAFT--do not distribute}    % These are ignored unless
\preprintfooter{} % 'preprint' option specified.

% Paper geometry
\special{papersize=8.5in,11in}
\setlength{\pdfpageheight}{\paperheight}
\setlength{\pdfpagewidth}{\paperwidth}

%% Make title
\maketitle

%% Abstract
\begin{abstract}
An abstract will appear here\dots
\end{abstract}

%% Introduction
\section{Introduction}
Introduction\dots

\subsection{Outline}

\section{Effect systems in practice}

\section{Algebraic effects and their handlers}

\section{Implementations}
Since their inception, several implementations of algebraic effect
handlers have appeared, many of which are implemented as libraries in
existing programming languages.
%
%% \citet{Swierstra2008}
%% provide an account of free monads for functional programmers.
%
Many of the library implementations of effect handlers include
implementations based on free monads~\cite{KammarLO13, KiselyovSS13,
  KiselyovI15, Brady13, WuSH14} which provide a natural basis for
implementing handlers. However, using free monads directly typically
leads to an inefficient implementation as free monads naïvely
materalise intermediate data constructors.

\citet{KammarLO13} provide an implementation of effect handlers in
Haskell using a continuation monad, which avoids materialising any
data constructors. \citet{WuS15} explain how it works, by taking
advantage of Glasgow Haskell Compiler's function fusion
optimisations.

\citet{SalehS16} implement effect handlers in Prolog by means of
elaboration into delimited control
operators~\cite{SchrijversDDW13}. Extensive use of delimited control
structures incurs a significant performance penalty. They recover
performance using a combination of partial evaluation and effect-based
rewrite rules.

The Idris effects library~\cite{Brady13,Brady14} takes advantage of dependent
types to provide effect handlers for a form of effects corresponding
to parameterised monads~\cite{Atkey09}.

I am aware of three languages that have been retrofitted with effect
handlers:
\begin{itemize}

\item Links~\cite{CooperLWY06} is a single source language for
  multi-tier web programming with an advanced type system which
  includes a linear type system and an effect system based on row
  polymorphism. Through a mild extension I have added effect handlers
  to Links~\cite{Hillerstrom15,HillerstromL16}. The implementation is
  based on a generalised abstract CEK
  machine~\cite{HillerstromL16}. In recent work, I extended the Links
  infrastructure with a compiler for effect
  handlers~\cite{HillerstromLS16}. The compiler interfaces with the
  Multicore OCaml backend in order to produce efficient native code.

\item Koka by \citet{Leijen14} is a functional web-oriented language
  which has recently been enriched with effect
  handlers~\cite{Leijen17}. It has a type-and-effect system which is
  also based on row polymorphism. Koka uses a novel compilation scheme
  which takes advantage of effect information in order to derive an
  efficient implementation of handlers on managed platforms such as
  .NET and JavaScript.

\item Multicore OCaml~\cite{DolanWSYM15} is an experimental branch of
  the OCaml compiler which extends OCaml with multicore
  capabilities. Multicore OCaml implements a restricted form of
  handlers which permit continuations to be invoked at most once. This
  design admits an efficient implementation, as continuations need
  never be copied, so they can simply be stored on the
  stack~\cite{BruggemanWD96}.

\end{itemize}
% The latter is particularly interesting as it is the first
% industrial-strength programming language with a focus on performance
% to adopt effect handlers.

I am aware of three languages that are specifically designed with
effect handlers in mind:

\begin{itemize}
\item Eff~\cite{BauerP15} is a strict language with Hindley-Milner
  type inference similar in spirit to ML, but extended with effect
  handlers.
%
  It includes a novel feature for supporting fresh generation of
  effects in order to support effects such as ML-style higher-order
  state (which has an operation for generating new references).

\item Frank~\cite{LindleyMM17} is a language with effect handlers but
  no separate notion of function: a function is but a special case of
  a handler. Frank has a bidirectional type and effect system with a
  novel form of effect polymorphism.
  
  %  in which
  % the programmer need never read or write effect variables. The effect
  % system of Frank can be viewed as implementing a form of row
  % polymorphism. 
% Unlike Links, but much like Koka~\cite{Leijen14},
%   Frank allows multiple occurrences of the same label in a row. In
%   contrast rows in Links are based on Remy's design in which
%   duplicates are not allowed, but negative information is.

\item Shonky~\cite{McBride16} amounts to a dynamically-typed variant
  of Frank. Though, handlers must be annotated with the names of the
  effects that they handle. The implementation of Shonky is based a
  variant of the CEK machine used in Links.
  % The main differences are that Shonky uses another
  % intermediate representation which gives a completely flat
  % continuation structure.
\end{itemize}

%% Main sections
\section{Generative effects}

\section{Efficient Effect Handlers}

\section{Scoped Effect Handlers}

\section{Conclusions and future work}

%% Bibliography
\nocite{*}
\bibliographystyle{abbrvnat}
\raggedright\bibliography{\jobname}

\end{document}